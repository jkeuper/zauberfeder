\section{High-Level Summary}
{\fullname} was tasked with performing an internal penetration test towards Offensive Security Labs.
An internal penetration test is a dedicated attack against internally connected systems. The focus of this
test is to perform attacks, similar to those of a hacker and attempt to infiltrate Offensive Security’s
internal lab systems – the THINC.local domain. \firstname’s overall objective was to evaluate the network,
identify systems, and exploit flaws while reporting the findings back to Offensive Security.
\\\\
When performing the internal penetration test, there were several alarming vulnerabilities that were
identified on Offensive Security’s network. When performing the attacks, {\firstname} was able to gain access
to multiple machines, primarily due to outdated patches and poor security configurations. During the
testing, {\firstname} had administrative level access to multiple systems. All systems were successfully
exploited and local access granted. Some machines were granted root/administrator access. These
systems as well as a brief description on how access was obtained are listed below:

\begin{itemize}
\item \ipA – Got in through \vulnxA
\ifdefined\gotB\item \ipB – Got in through \vulnxB
\ifdefined\gotC
\item \ipC – Got in through \vulnxC
\ifdefined\gotD
\item \ipD – Got in through \vulnxD
\ifdefined\gotE
\item \ipE – Got in through \vulnxE
\fi
\fi
\fi
\fi
\end{itemize}

\subsection{Recommendations}
{\firstname} recommends patching the vulnerabilities identified during the testing to ensure that an attacker
cannot exploit these systems in the future. One thing to remember is that these systems require
frequent patching and once patched, should remain on a regular patch program to protect additional
vulnerabilities that are discovered at a later date. Developing your own software is a big risk. Let you
developers be more security aware and have the software and source code audited by security
specialists.

\section{Methodologies}
{\firstname} utilized a widely adopted approach to performing penetration testing that is effective in testing
how well the Offensive Security Exam environments are secure. Below is a breakout of how {\firstname} was
able to identify and exploit the variety of systems and includes all individual vulnerabilities found.

\subsection{Information Gathering}
The information gathering portion of a penetration test focuses on identifying the scope of the
penetration test. During this penetration test, {\firstname} was tasked with exploiting the lab and exam
network. The specific IP addresses were:

\paragraph{Network:}
\ipaddresses

\subsection{Service Enumeration}
The service enumeration portion of a penetration test focuses on gathering information about what
services are alive on a system or systems. This is valuable for an attacker as it provides detailed
information on potential attack vectors into a system. Understanding what applications are running on
the system gives an attacker needed information before performing the actual penetration test. In
some cases, some ports may not be listed.

\begin{table}[h]
  %\begin{tabular}{|c|c|}
  \begin{tabularx}{\textwidth}{|l|X|}
    \hline
    \textbf{IP Address} & \textbf{Open ports} \\\hline
    \ipA                & \textbf{TCP:} \tcpportsA{}  \\
                        & \textbf{UDP:} \udpportsA{}  \\\hline
\ifdefined\gotB
    \ipB                & \textbf{TCP:} \tcpportsB{}  \\
                        & \textbf{UDP:} \udpportsB{}  \\\hline
\ifdefined\gotC
    \ipC                & \textbf{TCP:} \tcpportsC{}  \\
                        & \textbf{UDP:} \udpportsC{}  \\\hline
\ifdefined\gotD
    \ipD                & \textbf{TCP:} \tcpportsD{}  \\
                        & \textbf{UDP:} \udpportsD{}  \\\hline
\ifdefined\gotE
    \ipE                & \textbf{TCP:} \tcpportsE{}  \\
                        & \textbf{UDP:} \udpportsE{}  \\\hline
\fi
\fi
\fi
\fi
  \end{tabularx}
  \caption{Service enumeration}
\end{table}

\subsection{Penetration}
The penetration testing portions of the assessment focus heavily on gaining access to a variety of
systems. During this penetration test, {\firstname} was able to successfully gain access to 
{\machinecount} out of the {\machinecount} systems. On {\rootedcount} out of the {\machinecount} 
machines, {\firstname} gained root/administrator privileges.

